\resetproblem \begingroup % \jeolmheader
    \def\jeolmdate{19 октября 2018 г.}% пятница
    %\def\jeolmauthors{Виктория Журавлева}%
\jeolmheader \endgroup

\worksheet{Комбинаторика (разнобой)}

\begin{problems}

\item Таблицу размером $3\times3$ надо заполнить числами $-1$, $0$, $1$ так, чтобы суммы чисел в строках были одинаковыми. Сколькими способами это можно сделать? (Способы считаются различными, если различаются полученные таблицы. Все числа использовать не обязательно.) %831

\item В зале стоят шесть стульев в два ряда – по три стула в каждом, один ряд ровно за другим. В зал пришли шесть человек различного роста. Сколькими способами можно рассадить их так, чтобы каждый человек, сидящий в первом ряду, был ниже человека, сидящего за ним? %90

\item Сколькими способами можно нарисовать прямоугольник на клетчатом листе бумаги размером $m \times n$ клеток? (Например, на клетчатом листе $2\times 2$  можно нарисовать прямоугольник девятью различными способами). 

\item а) Сколько ожерелий можно составить из пяти одинаковых красных бусинок и двух одинаковых синих бусинок?
%3

б) Сколько существует различных наборов бусинок, из которых можно составить ровно два различных ожерелья?%сскк, ссккк, сскж, ссскж

\item Сколькими способами можно раскрасить колесо обозрения:

а) с $7$ кабинками в $3$ цвета; %\frac{3^7-3}{7}+3

б) c $10$ кабинками в $2$ цвета? %990/10+30/5+2/2+2=108

При раскраске не обязательно использовать все цвета.

\item а) Сколько существует различных игральных кубиков (на гранях кубика расставлены числа от $1$ до $6$)? %30

б) Та же задача для додекаэдра (числа от $1$ до $12$). %11!/5

\item На полке стоят $10$ различных книг. 

а) Сколькими способами их можно переставить так, чтобы ни одна книга не осталась на своем месте?

б) Докажите, что количество перестановок книг, при которых на месте остается ровно $4$ книги, больше $50 000$.

\item Троллейбусный билет имеет номер, состоящий из $6$ цифр. Билет считается счастливым, если сумма первых трех цифр равна сумме последних трех цифр. Найти количество счастливых билетов. %55252



\end{problems}