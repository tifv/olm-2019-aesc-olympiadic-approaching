\documentclass[a4paper,10pt]{article}
\usepackage[utf8]{inputenc}

% Конфигурация — общее {{{1

\usepackage{multicol}
\usepackage[svgnames]{xcolor}
\usepackage{graphicx}

\usepackage[lazy-figures]{jeolm}
\usepackage{jeolm-groups}

\usepackage[T2A]{fontenc}
\usepackage{anyfontsize}
\usepackage{amsmath}
\usepackage{amssymb}
\usepackage{upgreek}
\AtBeginDocument{\swapvar{phi}\swapvar{epsilon}}
\AtBeginDocument{\swapvar[up]{phi}\swapvar[up]{epsilon}}
\AtBeginDocument{\let\geq\geqslant\let\leq\leqslant}
\usepackage[russian]{babel}
\usepackage{parskip}
\pagestyle{empty}

% Конфигурация — страницы {{{1

\usepackage{geometry}
% Размер «логической» страницы
\geometry{a5paper,portrait,vmargin={2em,2em},hmargin={2em,2em}}
%\geometry{a6paper,landscape,vmargin={1.5em,1.5em},hmargin={1.5em,1.5em}}

%\usepackage{pgfpages}
% Размещение «логических» страниц на «физической»
%\pgfpagesuselayout{resize to}[a4paper]
%\pgfpagesuselayout{2 on 1}[a4paper,landscape]
%\pgfpagesuselayout{4 on 1}[a4paper,landscape]

\usepackage{hyperref}
\hypersetup{colorlinks,urlcolor=blue}
\def\maybephantomsection{\ifdefined\phantomsection\phantomsection\fi}

% Конфигурация — локальное {{{1

\renewcommand\jeolminstitution
  {СУНЦ МГУ, Олимпиадная математика}%
\renewcommand\jeolmdaterange
  {2018--2019}%
\def\jeolmgroupname{Догоняющие}%

% Конфигурация — размер шрифта {{{1

% Уменьшить основной размер шрифта
%\AtBeginDocument{\fontsize{9.00}{10.80}\selectfont}
% (по умолчанию 10.00 и 12.00 соответственно)

% }}}1


\begin{document}

\clearpage\resetproblem \begingroup % \jeolmheader
    \def\jeolmdate{24 октября 2018}% среда
    %\def\jeolmauthors{Витя Трещев}%
\jeolmheader \endgroup

\worksheet{Конструктивы}

\begin{problems}

\item
Нарисуйте фигуру, которую можно разрезать на~четыре фигурки, изображенные
слева, а~можно~--- на~пять фигурок, изображенных справа.
(Фигурки можно поворачивать.)
\begin{center}
    \jeolmfigure[width=0.1\linewidth]{pentomino-T}
\qquad
    \jeolmfigure[width=0.1\linewidth]{tetromino-T}
\end{center}

\item
Существуют~ли натуральные числа $m$ и~$n$, для которых верно равенство:
$(-2 a^n b^n)^m + (3 a^m b^m)^n = a^6 b^6$?

\item Можно ли разрезать квадрат на $2018$ прямоугольников так, что никакие два прямоугольника не имеют общих сторон? 

\item
Существуют~ли такие натуральные числа $a$, $b$, $c$, $d$, что
$a^3 + b^3 + c^3 + d^3 = 100^{100}$?

\item
На~шахматной доске $8 \times 8$ стоит кубик
(нижняя грань совпадает с~одной из~клеток доски).
Его прокатили по~доске, перекатывая через ребра, так, что кубик побывал на~всех
клетках (на~некоторых, возможно, несколько раз).
Могло~ли случиться, что одна из~его граней ни~разу не~лежала на~доске?

\item Существует ли такой выпуклый пятиугольник $ABCDE$, что все углы $ABD$, $BCE$, $CDA$, $DEB$ и $EAC$~--- тупые?

\item
На~плоскости нарисован черный квадрат.
Имеется семь квадратных плиток того~же размера.
Можно~ли расположить их на~плоскости так, чтобы они не~перекрывались и~чтобы
каждая плитка покрывала хотя~бы часть черного квадрата?

\item
Tреугольник разбили на~пять треугольников, ему подобных.
Bерно~ли, что исходный треугольник -- прямоугольный?

\item
Шесть отрезков таковы, что из~любых трех можно составить треугольник.
Bерно~ли, что из~этих отрезков можно составить тетраэдр?

\item
\subproblem
В~бесконечной последовательности бумажных прямоугольников площадь $n$-го
прямоугольника равна $n^2$.
Обязательно~ли можно покрыть ими плоскость?
Наложения допускаются.
\\
\subproblem
Дана бесконечная последовательность бумажных квадратов.
Обязательно~ли можно покрыть ими плоскость (наложения допускаются), если
известно, что для любого числа~$N$ найдется набор квадратов суммарной площади
больше~$N$?

\end{problems}



\end{document}

% Архив {{{1

\clearpage\resetproblem \begingroup % \jeolmheader
    \def\jeolmdate{19 октября 2018 г.}% пятница
    %\def\jeolmauthors{Виктория Журавлева}%
\jeolmheader \endgroup

\worksheet{Комбинаторика (разнобой)}

\begin{problems}

\item Таблицу размером $3\times3$ надо заполнить числами $-1$, $0$, $1$ так, чтобы суммы чисел в строках были одинаковыми. Сколькими способами это можно сделать? (Способы считаются различными, если различаются полученные таблицы. Все числа использовать не обязательно.) %831

\item В зале стоят шесть стульев в два ряда – по три стула в каждом, один ряд ровно за другим. В зал пришли шесть человек различного роста. Сколькими способами можно рассадить их так, чтобы каждый человек, сидящий в первом ряду, был ниже человека, сидящего за ним? %90

\item Сколькими способами можно нарисовать прямоугольник на клетчатом листе бумаги размером $m \times n$ клеток? (Например, на клетчатом листе $2\times 2$  можно нарисовать прямоугольник девятью различными способами). 

\item а) Сколько ожерелий можно составить из пяти одинаковых красных бусинок и двух одинаковых синих бусинок?
%3

б) Сколько существует различных наборов бусинок, из которых можно составить ровно два различных ожерелья?%сскк, ссккк, сскж, ссскж

\item Сколькими способами можно раскрасить колесо обозрения:

а) с $7$ кабинками в $3$ цвета; %\frac{3^7-3}{7}+3

б) c $10$ кабинками в $2$ цвета? %990/10+30/5+2/2+2=108

При раскраске не обязательно использовать все цвета.

\item а) Сколько существует различных игральных кубиков (на гранях кубика расставлены числа от $1$ до $6$)? %30

б) Та же задача для додекаэдра (числа от $1$ до $12$). %11!/5

\item На полке стоят $10$ различных книг. 

а) Сколькими способами их можно переставить так, чтобы ни одна книга не осталась на своем месте?

б) Докажите, что количество перестановок книг, при которых на месте остается ровно $4$ книги, больше $50 000$.

\item Троллейбусный билет имеет номер, состоящий из $6$ цифр. Билет считается счастливым, если сумма первых трех цифр равна сумме последних трех цифр. Найти количество счастливых билетов. %55252



\end{problems}
\clearpage\resetproblem \begingroup % \jeolmheader
    \def\jeolmdate{24 октября 2018}% среда
    %\def\jeolmauthors{Витя Трещев}%
\jeolmheader \endgroup

\worksheet{Конструктивы}

\begin{problems}

\item
Нарисуйте фигуру, которую можно разрезать на~четыре фигурки, изображенные
слева, а~можно~--- на~пять фигурок, изображенных справа.
(Фигурки можно поворачивать.)
\begin{center}
    \jeolmfigure[width=0.1\linewidth]{pentomino-T}
\qquad
    \jeolmfigure[width=0.1\linewidth]{tetromino-T}
\end{center}

\item
Существуют~ли натуральные числа $m$ и~$n$, для которых верно равенство:
$(-2 a^n b^n)^m + (3 a^m b^m)^n = a^6 b^6$?

\item Можно ли разрезать квадрат на $2018$ прямоугольников так, что никакие два прямоугольника не имеют общих сторон? 

\item
Существуют~ли такие натуральные числа $a$, $b$, $c$, $d$, что
$a^3 + b^3 + c^3 + d^3 = 100^{100}$?

\item
На~шахматной доске $8 \times 8$ стоит кубик
(нижняя грань совпадает с~одной из~клеток доски).
Его прокатили по~доске, перекатывая через ребра, так, что кубик побывал на~всех
клетках (на~некоторых, возможно, несколько раз).
Могло~ли случиться, что одна из~его граней ни~разу не~лежала на~доске?

\item Существует ли такой выпуклый пятиугольник $ABCDE$, что все углы $ABD$, $BCE$, $CDA$, $DEB$ и $EAC$~--- тупые?

\item
На~плоскости нарисован черный квадрат.
Имеется семь квадратных плиток того~же размера.
Можно~ли расположить их на~плоскости так, чтобы они не~перекрывались и~чтобы
каждая плитка покрывала хотя~бы часть черного квадрата?

\item
Tреугольник разбили на~пять треугольников, ему подобных.
Bерно~ли, что исходный треугольник -- прямоугольный?

\item
Шесть отрезков таковы, что из~любых трех можно составить треугольник.
Bерно~ли, что из~этих отрезков можно составить тетраэдр?

\item
\subproblem
В~бесконечной последовательности бумажных прямоугольников площадь $n$-го
прямоугольника равна $n^2$.
Обязательно~ли можно покрыть ими плоскость?
Наложения допускаются.
\\
\subproblem
Дана бесконечная последовательность бумажных квадратов.
Обязательно~ли можно покрыть ими плоскость (наложения допускаются), если
известно, что для любого числа~$N$ найдется набор квадратов суммарной площади
больше~$N$?

\end{problems}



% }}}1

\tableofcontents
\let\worksheetsave\worksheet
\def\worksheet#1{\maybephantomsection\addcontentsline{toc}{section}{#1}%
    \worksheetsave{#1}}
\clearpage\input{contents.tex} % auto-generated


% vim: set foldmethod=marker :%%%%%%%%%%%%%%%%%%%%%%%%%%%%%%%%%%%%%%%%%%%%%%%%%%
