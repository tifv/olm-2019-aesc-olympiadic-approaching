\documentclass[a4paper,10pt]{article}
\usepackage[utf8]{inputenc}

% Конфигурация — общее {{{1

\usepackage{multicol}
\usepackage[svgnames]{xcolor}
\usepackage{graphicx}

\usepackage[lazy-figures]{jeolm}
\usepackage{jeolm-groups}

\usepackage[T2A]{fontenc}
\usepackage{anyfontsize}
\usepackage{amsmath}
\usepackage{amssymb}
\usepackage{upgreek}
\AtBeginDocument{\swapvar{phi}\swapvar{epsilon}}
\AtBeginDocument{\swapvar[up]{phi}\swapvar[up]{epsilon}}
\AtBeginDocument{\let\geq\geqslant\let\leq\leqslant}
\usepackage[russian]{babel}
\usepackage{parskip}
\pagestyle{empty}

% Конфигурация — страницы {{{1

\usepackage{geometry}
% Размер «логической» страницы
\geometry{a5paper,portrait,vmargin={2em,2em},hmargin={2em,2em}}
%\geometry{a6paper,landscape,vmargin={1.5em,1.5em},hmargin={1.5em,1.5em}}

%\usepackage{pgfpages}
% Размещение «логических» страниц на «физической»
%\pgfpagesuselayout{resize to}[a4paper]
%\pgfpagesuselayout{2 on 1}[a4paper,landscape]
%\pgfpagesuselayout{4 on 1}[a4paper,landscape]

\usepackage{hyperref}
\hypersetup{colorlinks,urlcolor=blue}
\def\maybephantomsection{\ifdefined\phantomsection\phantomsection\fi}

% Конфигурация — локальное {{{1

\renewcommand\jeolminstitution
  {СУНЦ МГУ, Олимпиадная математика}%
\renewcommand\jeolmdaterange
  {2018--2019}%
\def\jeolmgroupname{Догоняющие}%

% Конфигурация — размер шрифта {{{1

% Уменьшить основной размер шрифта
\AtBeginDocument{\fontsize{10.00}{12.00}\selectfont}
% (по умолчанию 10.00 и 12.00 соответственно)

% }}}1


\begin{document}


\clearpage\resetproblem \begingroup % \jeolmheader
    \def\jeolmdate{28 ноября 2018}% среда
    %\def\jeolmauthors{Владислав Новиков}%
\jeolmheader \endgroup

\worksheet{Алгебраический разнобой}

\begin{problems}

\item Докажите, что если $a(a-b+c)<0$, то $b^2>4ac$.

\item Докажите, что если произведение положительных чисел $x$, $y$, $z$ равно 1, то $(2+x)(2+y)(2+z) \geqslant 27$.

\item Может ли при каком-то $n$ значение выражения $n^4 + 2n^3 + 2n^2 + 2n + 1$ быть точным квадратом натурального числа?

\item Докажите, что произведение $n$ последовательных натуральных чисел делится на $n!$.

\item Сумма нескольких положительных чисел равна 10, сумма их квадратов равна 20. Какое наименьшее значение может принимать сумма их кубов?


\item Верно ли, что у любого числа можно изменить не более одной цифры так, чтобы оно стало делиться на 11?

\item Есть $2n$ карточек с цифрой 1 и $n$ карточек с цифрой 2. Составьте из них 2 числа (используя все карточки) так, чтобы их разность была точным квадратом.

\item Решите в натуральных числах уравнение $a^3 + 1 = 3^n$.

\item Найдите все такие многочлены $P(x)$ с действительными коэффициентами, что многочлен $$(x+1) P(x-1) - (x-1)P(x)$$ является константой.


\item Приведите пример такого числа, что если его записать дважды подряд, получится точный квадрат.





\end{problems}
\clearpage\resetproblem \begingroup % \jeolmheader
    \def\jeolmdate{28 ноября 2018}% среда
    %\def\jeolmauthors{Владислав Новиков}%
\jeolmheader \endgroup

\worksheet{Алгебраический разнобой}

\begin{problems}

\item Докажите, что если $a(a-b+c)<0$, то $b^2>4ac$.

\item Докажите, что если произведение положительных чисел $x$, $y$, $z$ равно 1, то $(2+x)(2+y)(2+z) \geqslant 27$.

\item Может ли при каком-то $n$ значение выражения $n^4 + 2n^3 + 2n^2 + 2n + 1$ быть точным квадратом натурального числа?

\item Докажите, что произведение $n$ последовательных натуральных чисел делится на $n!$.

\item Сумма нескольких положительных чисел равна 10, сумма их квадратов равна 20. Какое наименьшее значение может принимать сумма их кубов?


\item Верно ли, что у любого числа можно изменить не более одной цифры так, чтобы оно стало делиться на 11?

\item Есть $2n$ карточек с цифрой 1 и $n$ карточек с цифрой 2. Составьте из них 2 числа (используя все карточки) так, чтобы их разность была точным квадратом.

\item Решите в натуральных числах уравнение $a^3 + 1 = 3^n$.

\item Найдите все такие многочлены $P(x)$ с действительными коэффициентами, что многочлен $$(x+1) P(x-1) - (x-1)P(x)$$ является константой.


\item Приведите пример такого числа, что если его записать дважды подряд, получится точный квадрат.





\end{problems}

\end{document}

% Архив {{{1
\clearpage\resetproblem \begingroup % \jeolmheader
    \def\jeolmdate{16 ноября 2018}% пятница
    %\def\jeolmauthors{Виктория Журавлева}%
\jeolmheader \endgroup

\worksheet{Рекуррентные соотношения в комбинаторике}

\begin{problems}

\item Сколько существует строк длины $10$, состоящих из нулей и единиц, таких, что никакие два нуля не стоят рядом?

\item Последовательность $a_n$ такова, что $a_1=4$, $a_2=25$. Найдите $a_{200},$ если для любого натурального $n$ справедливо равенство $a_{n+1}=a_n \cdot a_{n+2}$

%25

\item Функция $f(x)$ такова, что для всех значений $x$ выполняется равенство  $f(x + 1) = f(x) + 2x + 3$.  Известно, что  $f(0) = 1$.  Найдите $f(2012)$.

%2013^2

\item Существует ли $2005$ таких различных натуральных чисел, что сумма любых $2004$ из них делится на оставшееся число?

%1,2,3 дальше добавляем сумму

\item Сколько имеется разбиений отрезка длины $8$ на отрезки длины $1$, $2$ и $3$? (Разбиения, отличающиеся порядком следования отрезков, считаются различными.)

%81

\item Сколькими способами можно разменять купюру в $100$ рублей на монеты достоинством $1$, $2$ и $5$ рублей?

%b_n=b_{n-2}+1, нужное 4+b_{100}+b_{95}+...+b_{10}
%541

\item Сколько слов длины $10$ можно составить из букв <<a>>, <<б>>, <<в>>, так чтобы буквы <<a>> и <<б>> не стояли рядом?

%x_n=2x_{n-1}+x_{n-2}, x_1=3, x_2=7

\item Петя выписывает все возможные $2018$-буквенные слова, состоящие только из букв <<a>>, <<б>>, <<в>>. В скольких из них <<a>> встречается четное количество раз?

%\frac{3^{2018}+1}{2}


\item Кузнечик прыгает по вершинам правильного треугольника $ABC$, прыгая каждый раз в одну из соседних вершин. Сколькими способами он может попасть из вершины $A$ обратно в вершину A за $10$ прыжков?

%a_n=a_{n-1}+2a_{n-2}

\item Сколькими способами можно выложить прямоугольник размера $10\times 3$ доминошками размера $1\times 2$?

%a_n=4a_{n-2}-A_{n-4}, A_2=3, A_4=11





\end{problems}
\clearpage\resetproblem \begingroup % \jeolmheader
    \def\jeolmdate{19 октября 2018 г.}% пятница
    %\def\jeolmauthors{Виктория Журавлева}%
\jeolmheader \endgroup

\worksheet{Комбинаторика (разнобой)}

\begin{problems}

\item Таблицу размером $3\times3$ надо заполнить числами $-1$, $0$, $1$ так, чтобы суммы чисел в строках были одинаковыми. Сколькими способами это можно сделать? (Способы считаются различными, если различаются полученные таблицы. Все числа использовать не обязательно.) %831

\item В зале стоят шесть стульев в два ряда – по три стула в каждом, один ряд ровно за другим. В зал пришли шесть человек различного роста. Сколькими способами можно рассадить их так, чтобы каждый человек, сидящий в первом ряду, был ниже человека, сидящего за ним? %90

\item Сколькими способами можно нарисовать прямоугольник на клетчатом листе бумаги размером $m \times n$ клеток? (Например, на клетчатом листе $2\times 2$  можно нарисовать прямоугольник девятью различными способами). 

\item а) Сколько ожерелий можно составить из пяти одинаковых красных бусинок и двух одинаковых синих бусинок?
%3

б) Сколько существует различных наборов бусинок, из которых можно составить ровно два различных ожерелья?%сскк, ссккк, сскж, ссскж

\item Сколькими способами можно раскрасить колесо обозрения:

а) с $7$ кабинками в $3$ цвета; %\frac{3^7-3}{7}+3

б) c $10$ кабинками в $2$ цвета? %990/10+30/5+2/2+2=108

При раскраске не обязательно использовать все цвета.

\item а) Сколько существует различных игральных кубиков (на гранях кубика расставлены числа от $1$ до $6$)? %30

б) Та же задача для додекаэдра (числа от $1$ до $12$). %11!/5

\item На полке стоят $10$ различных книг. 

а) Сколькими способами их можно переставить так, чтобы ни одна книга не осталась на своем месте?

б) Докажите, что количество перестановок книг, при которых на месте остается ровно $4$ книги, больше $50 000$.

\item Троллейбусный билет имеет номер, состоящий из $6$ цифр. Билет считается счастливым, если сумма первых трех цифр равна сумме последних трех цифр. Найти количество счастливых билетов. %55252



\end{problems}
\clearpage\resetproblem \begingroup % \jeolmheader
    \def\jeolmdate{24 октября 2018}% среда
    %\def\jeolmauthors{Витя Трещев}%
\jeolmheader \endgroup

\worksheet{Конструктивы}

\begin{problems}

\item
Нарисуйте фигуру, которую можно разрезать на~четыре фигурки, изображенные
слева, а~можно~--- на~пять фигурок, изображенных справа.
(Фигурки можно поворачивать.)
\begin{center}
    \jeolmfigure[width=0.1\linewidth]{pentomino-T}
\qquad
    \jeolmfigure[width=0.1\linewidth]{tetromino-T}
\end{center}

\item
Существуют~ли натуральные числа $m$ и~$n$, для которых верно равенство:
$(-2 a^n b^n)^m + (3 a^m b^m)^n = a^6 b^6$?

\item Можно ли разрезать квадрат на $2018$ прямоугольников так, что никакие два прямоугольника не имеют общих сторон? 

\item
Существуют~ли такие натуральные числа $a$, $b$, $c$, $d$, что
$a^3 + b^3 + c^3 + d^3 = 100^{100}$?

\item
На~шахматной доске $8 \times 8$ стоит кубик
(нижняя грань совпадает с~одной из~клеток доски).
Его прокатили по~доске, перекатывая через ребра, так, что кубик побывал на~всех
клетках (на~некоторых, возможно, несколько раз).
Могло~ли случиться, что одна из~его граней ни~разу не~лежала на~доске?

\item Существует ли такой выпуклый пятиугольник $ABCDE$, что все углы $ABD$, $BCE$, $CDA$, $DEB$ и $EAC$~--- тупые?

\item
На~плоскости нарисован черный квадрат.
Имеется семь квадратных плиток того~же размера.
Можно~ли расположить их на~плоскости так, чтобы они не~перекрывались и~чтобы
каждая плитка покрывала хотя~бы часть черного квадрата?

\item
Tреугольник разбили на~пять треугольников, ему подобных.
Bерно~ли, что исходный треугольник -- прямоугольный?

\item
Шесть отрезков таковы, что из~любых трех можно составить треугольник.
Bерно~ли, что из~этих отрезков можно составить тетраэдр?

\item
\subproblem
В~бесконечной последовательности бумажных прямоугольников площадь $n$-го
прямоугольника равна $n^2$.
Обязательно~ли можно покрыть ими плоскость?
Наложения допускаются.
\\
\subproblem
Дана бесконечная последовательность бумажных квадратов.
Обязательно~ли можно покрыть ими плоскость (наложения допускаются), если
известно, что для любого числа~$N$ найдется набор квадратов суммарной площади
больше~$N$?

\end{problems}


\clearpage\resetproblem \begingroup % \jeolmheader
    \def\jeolmdate{07 ноября 2018г.}% среда
    \def\jeolmauthors{}%
\jeolmheader \endgroup

\worksheet{Соответствия}

\begin{problems}

\item Рассматриваются всевозможные треугольники, имеющие целочисленные стороны и периметр которых равен $2000$, а также всевозможные треугольники, имеющие целочисленные стороны и периметр которых равен $2003$. Каких треугольников больше? 

\item Каких чисел больше среди всех чисел от $100$ до $999$: тех, у которых средняя цифра больше обеих крайних, или тех, у которых средняя цифра меньше обеих крайних?

\item Полоска $1\times 10$ разбита на единичные квадраты. В квадраты записывают числа $1, 2, \dots, 10$. Сначала в один какой-нибудь квадрат записывают число $1$, затем число $2$ записывают в один из соседних квадратов, затем число $3$ --- в один из соседних с уже занятыми и т. д. (произвольными являются выбор первого квадрата и выбор соседа на каждом шагу). Сколькими способами это можно проделать?

\item \subproblem Автобусные билеты имеют шестизначные номера от $000000$ до $999999$. Билет называется счастливым, если сумма первых трех его цифр его номера равна сумме последних трех. Является ли четным число счастливых билетов?

\subproblem Является ли чётным число всех $20$-значных натуральных чисел, не содержащих в записи нулей и делящихся на $101$?

\item Дана шахматная доска. Ее горизонтали перенумерованы числами от $1$ до $8$, а~вертикали обозначены латинскими буквами от $a$ до $h$. Рассматриваются покрытия доски доминошками. Каких разбиений больше~--- тех, которые содержат доминошку $a1-a2$, или тех, которые содержат доминошку $b2-b3$?

\item Рассмотрим всевозможные графы на $n$ пронумерованных вершинах. Каких графов среди них больше --- связных или несвязных?

\item Дан выпуклый $n$-угольник такой, что никакие три его диагонали не пересекаются в одной точке. Найдите количество точек пересечения диагоналей данного многоугольника (не являющихся вершинами многоугольника).


\item \subproblem Докажите, что количество разбиений числа $n$ в сумму не более чем $k$ слагаемых, равно количеству разбиений числа $n$ в сумму слагаемых, не превосходящих $k$.

\subproblem Докажите, что количество разбиений числа $n$, равно количеству разбиений числа $2n$ в сумму ровно $n$ слагаемых.

\item Доказать, что суммарное количество цифр в десятичной записи чисел $1, 2, \ldots, 10^k$ равно суммарному количеству нулей в десятичной записи чисел $1, 2, \dots, 10^{k + 1}$.{\sloppy\par}

\item На окружности отмечено $2N$ точек ($N$ --- натуральное число). Известно, что через любую точку внутри окружности проходит не более двух хорд с концами в отмеченных точках. Назовем паросочетанием такой набор из $N$ хорд с концами в отмеченных точках, что каждая отмеченная точка является концом ровно одной из этих хорд. Назовём паросочетание чётным, если количество точек, в которых пересекаются его хорды, чётно, и нечётным иначе. Найдите разность между количеством чётных и нечётных паросочетаний.

\end{problems}

% }}}1

\tableofcontents
\let\worksheetsave\worksheet
\def\worksheet#1{\maybephantomsection\addcontentsline{toc}{section}{#1}%
    \worksheetsave{#1}}
\clearpage\input{contents.tex} % auto-generated


% vim: set foldmethod=marker :%%%%%%%%%%%%%%%%%%%%%%%%%%%%%%%%%%%%%%%%%%%%%%%%%%
