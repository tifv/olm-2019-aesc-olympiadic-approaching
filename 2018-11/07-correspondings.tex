\resetproblem \begingroup % \jeolmheader
    \def\jeolmdate{07 ноября 2018г.}% среда
    \def\jeolmauthors{}%
\jeolmheader \endgroup

\worksheet{Соответствия}

\begin{problems}

\item Рассматриваются всевозможные треугольники, имеющие целочисленные стороны и периметр которых равен $2000$, а также всевозможные треугольники, имеющие целочисленные стороны и периметр которых равен $2003$. Каких треугольников больше? 

\item Каких чисел больше среди всех чисел от $100$ до $999$: тех, у которых средняя цифра больше обеих крайних, или тех, у которых средняя цифра меньше обеих крайних?

\item Полоска $1\times 10$ разбита на единичные квадраты. В квадраты записывают числа $1, 2, \dots, 10$. Сначала в один какой-нибудь квадрат записывают число $1$, затем число $2$ записывают в один из соседних квадратов, затем число $3$ --- в один из соседних с уже занятыми и т. д. (произвольными являются выбор первого квадрата и выбор соседа на каждом шагу). Сколькими способами это можно проделать?

\item \subproblem Автобусные билеты имеют шестизначные номера от $000000$ до $999999$. Билет называется счастливым, если сумма первых трех его цифр его номера равна сумме последних трех. Является ли четным число счастливых билетов?

\subproblem Является ли чётным число всех $20$-значных натуральных чисел, не содержащих в записи нулей и делящихся на $101$?

\item Дана шахматная доска. Ее горизонтали перенумерованы числами от $1$ до $8$, а~вертикали обозначены латинскими буквами от $a$ до $h$. Рассматриваются покрытия доски доминошками. Каких разбиений больше~--- тех, которые содержат доминошку $a1-a2$, или тех, которые содержат доминошку $b2-b3$?

\item Рассмотрим всевозможные графы на $n$ пронумерованных вершинах. Каких графов среди них больше --- связных или несвязных?

\item Дан выпуклый $n$-угольник такой, что никакие три его диагонали не пересекаются в одной точке. Найдите количество точек пересечения диагоналей данного многоугольника (не являющихся вершинами многоугольника).


\item \subproblem Докажите, что количество разбиений числа $n$ в сумму не более чем $k$ слагаемых, равно количеству разбиений числа $n$ в сумму слагаемых, не превосходящих $k$.

\subproblem Докажите, что количество разбиений числа $n$, равно количеству разбиений числа $2n$ в сумму ровно $n$ слагаемых.

\item Доказать, что суммарное количество цифр в десятичной записи чисел $1, 2, \ldots, 10^k$ равно суммарному количеству нулей в десятичной записи чисел $1, 2, \dots, 10^{k + 1}$.{\sloppy\par}

\item На окружности отмечено $2N$ точек ($N$ --- натуральное число). Известно, что через любую точку внутри окружности проходит не более двух хорд с концами в отмеченных точках. Назовем паросочетанием такой набор из $N$ хорд с концами в отмеченных точках, что каждая отмеченная точка является концом ровно одной из этих хорд. Назовём паросочетание чётным, если количество точек, в которых пересекаются его хорды, чётно, и нечётным иначе. Найдите разность между количеством чётных и нечётных паросочетаний.

\end{problems}