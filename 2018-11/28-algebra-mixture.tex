\resetproblem \begingroup % \jeolmheader
    \def\jeolmdate{28 ноября 2018 г.}% среда
    %\def\jeolmauthors{Владислав Новиков}%
\jeolmheader \endgroup

\worksheet{Алгебраический разнобой}

\begin{problems}

\item Докажите, что если $a(a-b+c)<0$, то $b^2>4ac$.

\item Докажите, что если произведение положительных чисел $x$, $y$, $z$ равно 1, то $(2+x)(2+y)(2+z) \geqslant 27$.

\item Может ли при каком-то $n$ значение выражения $n^4 + 2n^3 + 2n^2 + 2n + 1$ быть точным квадратом натурального числа?

\item Докажите, что произведение $n$ последовательных натуральных чисел делится на $n!$.

\item Сумма нескольких положительных чисел равна 10, сумма их квадратов равна 20. Какое наименьшее значение может принимать сумма их кубов?


\item Верно ли, что у любого числа можно изменить не более одной цифры так, чтобы оно стало делиться на 11?

\item Есть $2n$ карточек с цифрой 1 и $n$ карточек с цифрой 2. Составьте из них 2 числа (используя все карточки) так, чтобы их разность была точным квадратом.

\item Решите в натуральных числах уравнение $a^3 + 1 = 3^n$.

\item Найдите все такие многочлены $P(x)$ с действительными коэффициентами, что многочлен $$(x+1) P(x-1) - (x-1)P(x)$$ является константой.


\item Приведите пример такого числа, что если его записать дважды подряд, получится точный квадрат.





\end{problems}