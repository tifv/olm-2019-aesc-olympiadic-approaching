\resetproblem \begingroup % \jeolmheader
    \def\jeolmdate{30 ноября 2018 г.}% пятница
    \def\jeolmauthors{Юлий Тихонов}%
\jeolmheader \endgroup

\worksheet{Гидроксид геометрии}

%\subsection*{Теория}

В~треугольнике $ABC$ отмечены центр описанной окружности $O$
и~ортоцентр (точка пересечения высот) $H$.

\begingroup
    \let\vect\overline

\claim{Лемма 1}
Прямые $AO$ и $AH$ переходят друг в друга при симметрии угла $A$.

\begin{problems}

\item
\subproblem
Докажите лемму 1 для остроугольного треугольника.
\\
\subproblem
Докажите лемму 1 для тупоугольного треугольника с тупым углом~$A$.
\\
\subproblem
Докажите лемму 1 для тупоугольного треугольника с острым углом $A$.

\end{problems}

\claim{Лемма 2}
Верно векторное равенство $\vect{OH} = \vect{OA} + \vect{OB} + \vect{OC}$.

\begin{problems}

\item
\subproblem
Докажите, что $\vect{OB} + \vect{OC} \perp BC$.
\\
\subproblem
Докажите, что $\vect{OA} + \vect{OB} + \vect{OC} - \vect{OH} \perp BC$.
\\
\subproblem
Докажите лемму 2.

\item
\subproblem
Пусть $M$~--- середина $BC$.
Докажите, что $\vect{AH} = 2 \cdot \vect{OM}$.
\\
\subproblem
Пусть фиксированы точки $B$ и $C$ и окружность, через них проходящая, а точка $A$ бегает по этой окружности.
Найдите кривую, по которой бегает $H$.

\end{problems}

И ещё.

\begin{problems}

\item
\subproblem
Точку $H$ отразили относительно стороны $BC$ и получили точку $A_0$.
Докажите, что $A_0$ лежит на описанной окружности $ABC$.
\\
\subproblem
Точку $H$ отразили относительно середины стороны $BC$ и получили точку $A_1$.
Докажите, что $A_1$ лежит на описанной окружности $ABC$, причем $AA_1$~--- диаметр.

\end{problems}

%\subsection*{Практика}
И задачи.

\begin{problems}

\item
Восстановите треугольник по~данным $A$, $O$ и~$H$.
(Сколько решений в зависимости от~расположения точек?)

\item
Проведем высоты $BB_1$ и $CC_1$.
Докажите, что прямая $AO$ содержит высоту треугольника $AB_1C_1$, а прямая $AH$ содержит центр описанной окружности этого треугольника.

\item
В~остроугольном треугольнике $ABC$ проведена биссектриса~$AD$.
Перпендикуляр, опущенный из~$B$ на~прямую~$AD$, пересекает описанную окружность треугольника $ABD$ в~точке~$E$, отличной от~$B$.
Докажите, что точки $A$, $E$ и~центр описанной окружности~$O$ треугольника $ABC$ лежат на~одной прямой.

%\item
%\subproblem
%Прямые $a$ и~$b$ пересекаются в~точке~$L$.
%Прямая~$a$ пересекает окружность~$\omega$ в~точках $A_1$ и~$A_2$,
%а~$b$ пересекает $\omega$ в~точках $B_1$ и~$B_2$.
%Пусть $R_1$~--- центр описанной окружности в~треугольнике $L A_1 B_1$.
%Докажите, что прямая~$L R_1$~--- высота в~треугольнике $L A_2 B_2$.
%\\
%\subproblem
%В~условиях предыдущего пункта пусть $O$~--- центр окружности~$\omega$.
%Докажите, что $L R_1 = O R_2$, где $R_2$ определяется аналогично $R_1$.

\item
Найдите углы остроугольного треугольника $ABC$, если известно, что его
биссектриса~$AD$ равна стороне~$AC$ и~перпендикулярна отрезку~$OH$, где $O$~---
центр описанной окружности, $H$~--- точка пересечения высот треугольника $ABC$.

\end{problems}

%Кстати, почему листочек так называется?

\endgroup % \def\vect
