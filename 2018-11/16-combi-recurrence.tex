\resetproblem \begingroup % \jeolmheader
    \def\jeolmdate{16 ноября 2018}% пятница
    %\def\jeolmauthors{Виктория Журавлева}%
\jeolmheader \endgroup

\worksheet{Рекуррентные соотношения в комбинаторике}

\begin{problems}

\item Сколько существует строк длины $10$, состоящих из нулей и единиц, таких, что никакие два нуля не стоят рядом?

\item Последовательность $a_n$ такова, что $a_1=4$, $a_2=25$. Найдите $a_{200},$ если для любого натурального $n$ справедливо равенство $a_{n+1}=a_n \cdot a_{n+2}$

%25

\item Функция $f(x)$ такова, что для всех значений $x$ выполняется равенство  $f(x + 1) = f(x) + 2x + 3$.  Известно, что  $f(0) = 1$.  Найдите $f(2012)$.

%2013^2

\item Существует ли $2005$ таких различных натуральных чисел, что сумма любых $2004$ из них делится на оставшееся число?

%1,2,3 дальше добавляем сумму

\item Сколько имеется разбиений отрезка длины $8$ на отрезки длины $1$, $2$ и $3$? (Разбиения, отличающиеся порядком следования отрезков, считаются различными.)

%81

\item Сколькими способами можно разменять купюру в $100$ рублей на монеты достоинством $1$, $2$ и $5$ рублей?

%b_n=b_{n-2}+1, нужное 4+b_{100}+b_{95}+...+b_{10}
%541

\item Сколько слов длины $10$ можно составить из букв <<a>>, <<б>>, <<в>>, так чтобы буквы <<a>> и <<б>> не стояли рядом?

%x_n=2x_{n-1}+x_{n-2}, x_1=3, x_2=7

\item Петя выписывает все возможные $2018$-буквенные слова, состоящие только из букв <<a>>, <<б>>, <<в>>. В скольких из них <<a>> встречается четное количество раз?

%\frac{3^{2018}+1}{2}


\item Кузнечик прыгает по вершинам правильного треугольника $ABC$, прыгая каждый раз в одну из соседних вершин. Сколькими способами он может попасть из вершины $A$ обратно в вершину A за $10$ прыжков?

%a_n=a_{n-1}+2a_{n-2}

\item Сколькими способами можно выложить прямоугольник размера $10\times 3$ доминошками размера $1\times 2$?

%a_n=4a_{n-2}-A_{n-4}, A_2=3, A_4=11





\end{problems}